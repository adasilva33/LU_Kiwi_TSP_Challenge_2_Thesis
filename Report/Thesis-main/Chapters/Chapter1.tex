\chapter{Introduction}
\label{Chapter1}
\lhead{Chapter 1. \emph{Introduction}}

%\todo{Set the scenes. Explain why you are doing this work and why the problem being solved is difficult. Most importantly you should clearly explain what the aims and objectives of your work are.}

%\todo{Structure of the thesis. Academic publications produced (if any), including any achievements/highlights}

\section{Background}

The number of flight connections keep increasing every year \cite{statista_flights_year}, more than 38 million flights have been scheduled in 2023 - therefore, creating a challenge for traveler's to find the best and cheapest flight connections for their specific journey, especially when one has to visit a big number of cities.
Consequently, travel agencies have deployed online trip planner algorithms in order to find flights connection that match the traveler's requirements. Example of these are, Google Flights, OpenFlights.org, Skyscanner, Kayak and Kiwi.com.

These agencies have launched different challenges to create and build powerful trip planner algorithms. For instance, as mentionned in \cite{reinforcement_learning_yaro}, OpenFlights.org launched the Air Travelling Salesman project. Furthermore, Kiwi.com has launched a project in 2017, called Traveling Salesman Challenge, where the current algorithm used by Kiwi.com was developed. In 2018, Kiwi.com launched a new challenge, the Traveling Salesman Problem 2.0 which is the focus of this study.

The given problem is a variant of the Traveling Salesman Problem. It can be characteristed as a generalised, assymetric and time dependant TSP.
A traveler has to visit a list of areas, one per day, given a starting airport and all the possible flight connections between these areas at different days. The goal is to determine what is the cheapest flights connection for the traveler to come back to the starting area. Regarding the number of possible journeys, solving this problem by exploring every single potential solution is impossible. This is why a heuristic approach is often used to solve such TSP problem. In this paper, the Kiwi.com challenge is solved using a Monte Carlo Tree Search.
\section{Research objectives}
\label{section:research_obj}
The goals of this dissertation are:
\begin{itemize}
    \item The implementation of a Python innovative solution to solve the Kiwi.com Traveling Salesman problem 2.0 with no focus on the time limit.
    \item Focus on instances ($I_1,\ldots I_8$) that represent more realistic scenarios.
    \item Try to find better solutions than the state of the art for the considered instances.

\end{itemize}

\section{Academic publication}
\begin{itemize}
    \item International Conference on Computer, Control, Electrical, and Electronics Engineering (ICCCEEE)
\end{itemize}

\section{Dissertation structure}

The dissertation is structured as follow:
\begin{itemize}
    \item Section \ref{Chapter2} is the litterature review where the Air Travel optimisations problem are introduced, TSP and its variants are redefined and finally the Monte Carlo Tree Search and an example are presented.
    \item Section \ref{Chapter3} is the problem and instances description to highlight the problem complexity in detail.
    \item Section \ref{Chapter4} is the methodology of our algorithm implementation, where we explain the code's structure, explain the general flow of the algorithm.
    \item Section \ref{Chapter5} is the result and performance of our implementation compared to the state of the art solution and further analysises regarding the MCTS' parameters.
\end{itemize}

